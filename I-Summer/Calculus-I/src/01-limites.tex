\chapter{Límites}

\section{Definición intuitiva}

El límite de una función $f$ 
en un punto $x_0$ es un número $L$ que se aproxima a 
$f(x_0)$ cuando $x$ se acerca a $x_0$. Esto quiere decir 
que el límite permite investigar el comportamiento de una 
función en un determinado punto. \\

Una forma sencilla de expresar esto es mediante el uso de una 
tabla de valores:

\ex{} {
    Por ejemplo, si $f(x) = x^2$ y $x_0 = 2$, 
    podemos obtener una aproximación de $f(x_0)$ mediante la 
    siguiente tabla:

    \begin{center}
        \begin{tabular}{c|c|c|c|c|c|c|c|c|c|c|c}
            $x$ & $1$ & $1.5$ & $1.9$ & $1.99$ & $1.999$ & $2$ & $2.001$ & $2.01$ & $2.1$ & $2.5$ & $3$ \\
            \hline
            $f(x)$ & $1$ & $2.25$ & $3.61$ & $3.9601$ & $3.996001$ & $4$ & $4.004001$ & $4.0401$ & $4.41$ & $6.25$ & $9$ \\
        \end{tabular}
    \end{center}

    % TODO: Add figure

    Podemos observar que el valor de $f(x)$ se acerca a $4$
    cuando $x$ se acerca a $2$. Por lo tanto, podemos decir
    que el límite de $f$ en $x_0 = 2$ es $L = 4$, y se 
    representa mediante la siguiente notación:

    \[
        \lim_{x \to 2} f(x) = 4
    \]
}

\section{Límites laterales}

El límite lateral de una función $f$ en un punto $x_0$ es el 
valor que toma $f$ cuando $x$ se acerca a $x_0$ desde un 
lado. Por ejemplo, si $x_0 = 2$ y $x$ se acerca a $x_0$ desde 
el lado izquierdo, entonces el límite lateral izquierdo de 
$f$ en $x_0$ es el valor que toma $f$ cuando $x$ se acerca a 
$x_0$ desde el lado izquierdo. De manera similar, el límite 
lateral derecho de $f$ en $x_0$ es el valor que toma $f$ 
cuando $x$ se acerca a $x_0$ desde el lado derecho. \\

La notación para el límite lateral izquierdo de $f$ en 
$x_0$ es: 

\[
    \lim_{x \to 2^-} f(x)
\]

y la notación para el límite lateral derecho de $f$ en 
$x_0$ es:

\[
    \lim_{x \to 2^+} f(x)
\]

Si los límites laterales de una función en un punto son 
iguales, entonces el límite de la función en ese punto es igual a 
cualquiera de los límites laterales. Esto se puede expresar de la 
siguiente manera:

\[
    \lim_{x \to x_0} f(x) = L \iff \lim_{x \to x_0^+} f(x) = L \wedge \lim_{x \to x_0^-} f(x) = L
\]

Dando a entender que si los límites laterales no son iguales, entonces
el límite de la función en ese punto no existe. \\ 

\ex{} {
    Si $g(x) = \frac{1}{x}$ y $x_0 = 0$, podemos 
    obtener una aproximación de $g(x_0)$ mediante la siguiente 
    tabla:

    \begin{center}
        \begin{tabular}{c|c|c|c|c|c|c|c|c|c|c|c}
            $x$ & $-3$ & $-2$ & $-1$ & $-0.5$ & $-0.1$ & $0$ & $0.1$ & $0.5$ & $1$ & $2$ & $3$ \\
            \hline
            $g(x)$ & $-\frac{1}{3}$ & $-\frac{1}{2}$ & $-1$ & $-2$ & $-10$ & & $10$ & $2$ & $1$ & $\frac{1}{2}$ & $\frac{1}{3}$ \\
        \end{tabular}
    \end{center}

    Gracias a la tabla, se puede observar que el valor de $g(x)$ 
    se acerca a $-\infty$ cuando $x$ se acerca a $0$ desde el 
    lado izquierdo, y se acerca a $\infty$ cuando $x$ se acerca a 
    $0$ desde el lado derecho. Por lo tanto, podemos decir que el
    límite lateral izquierdo de la función 
    $g(x) = \frac{1}{x}$ en $x_0 = 1$ es $-\infty$ y el límite lateral 
    derecho de la función $g(x) = \frac{1}{x}$ en $x_0 = 1$ es 
    $\infty$. \\

    En este caso, podemos observar que los límites laterales de 
    $g$ en $x_0 = 0$ son diferentes, por lo tanto, el límite de 
    $g$ en $x_0 = 0$ no existe.
}

\section{Definición formal}

\dfn{} {
    El límite de una función $f$ en un punto $x_0$ es un número 
    $L$ tal que, para todo $\epsilon > 0$, existe un $\delta > 0$ tal que, si 
    $0 < |x - x_0| < \delta$, entonces $|f(x) - L| < \epsilon$.

    \[
        \lim_{x \to x_0} f(x) = L \iff \forall \epsilon > 0 \text{ } \exists \delta > 0 \text{ } \forall x \in \mathbb{R} \text{ tal que } 0 < |x - x_0| < \delta \implies |f(x) - L| < \epsilon 
    \]

    Donde $\epsilon$ es la precisión que queremos alcanzar y 
    $\delta$ es la distancia mínima entre $x$ y $x_0$. 
}

\ex{} {
    Esto se puede demostar utilizando la función $f(x) = \frac{2x^2-x-1}{x-1}$,
    que tiene un límite en $x_0 = 1$. \\

    Usando la definición del límite de una función, se debe demostrar que 
    \[
      \lim_{x \to 1} f(x) = 3  
    \]

    Sea $\epsilon$ un número positivo arbitrario. Se debe encontrar un 
    $\delta > 0$ tal que, si $0 < |x - 1| < \delta$, entonces
    $|f(x) - 3| < \epsilon$. \\

    Para ello, se toma la siguiente expresión:

    \[
        \begin{align*}
            \left|\frac{2x^2-x-1}{x-1} - 3\right| < \epsilon &\iff \left|\frac{(2x+1)(x-1)}{x-1} - 3\right| < \epsilon \\
            &\iff |(2x+1) - 3| < \epsilon \\
            &\iff |2x - 2| < \epsilon \\
            &\iff |2(x-1)| < \epsilon \\
            &\iff 2|x-1| < \epsilon \\
            &\iff |x-1| < \frac{\epsilon}{2} \wedge x \neq 1
        \end{align*}
    \]

    Comparando con la definición del límite, se tiene que $\delta = \frac{\epsilon}{2}$. \\

    Dado $\epsilon > 0$, existe $\delta > 0$ tal que:

    \[
        \begin{align*}
            0 < |x - 1| < \delta &\implies |x-1| < \delta \wedge x \neq 1 \\
            &\implies |x-1| < \frac{\epsilon}{2} \wedge x \neq 1 \\
            &\implies |2x - 2| < \epsilon \wedge x \neq 1 \\
            &\implies |(2x+1) - 3| < \epsilon \wedge x \neq 1 \\
            &\implies \left|\frac{(2x+1)(x-1)}{x-1} - 3\right| < \epsilon \\
            &\implies \left|\frac{2x^2-x-1}{x-1} - 3\right| < \epsilon
        \end{align*}
    \]

    De esta forma, se ha demostrado que el límite de $f$ en $x_0 = 1$ es $L = 3$.
}

\section{Propiedades}

\thm{Propiedades de los límites} {
    Si $c$ es una constante y $\lim_{x \to a} f(x) = L$ y $\lim_{x \to a} g(x) = M$, entonces:

    \begin{enumerate}
        \item $\lim_{x \to a} f(x) = L$
        \item $\lim_{x \to a} c = c$
        \item $\lim_{x \to a} x = a$
        \item $\lim_{x \to a} f(x) + g(x) = \lim_{x \to a} f(x) + \lim_{x \to a} g(x) = L + M$
        \item $\lim_{x \to a} f(x) - g(x) = \lim_{x \to a} f(x) - \lim_{x \to a} g(x) = L - M$ 
        \item $\lim_{x \to a} f(x)g(x) = \lim_{x \to a} f(x) \cdot \lim_{x \to a} g(x) = L \cdot M$
        \item $\lim_{x \to a} \frac{f(x)}{g(x)} = \frac{\lim_{x \to a} f(x)}{\lim_{x \to a} g(x)} = \frac{L}{M}, M \neq 0$
        \item $\lim_{x \to a} c \cdot f(x) = c \cdot \lim_{x \to a} f(x) = c \cdot L$
        \item $\lim_{x \to a} [f(x)]^n = \left[\lim_{x \to a} f(x)\right]^n = L^n, n \in \mathbb{N}$
        \item $\lim_{x \to a} x^n = a^n, n \in \mathbb{N}$
        \item $\lim_{x \to a} \sqrt[n]{f(x)} = \sqrt[n]{\lim_{x \to a} f(x)} = \sqrt[n]{L}, n \in \mathbb{N}$
        \item $\lim_{x \to a} \sqrt[n]{x} = \sqrt[n]{a}, n \in \mathbb{N}$
    \end{enumerate}
}

El propósito de conocer estas propiedades es que si se tiene 
una función polinomial, racional o radical y $a$ pertenece al dominio de la función, 
entonces se puede calcular el límite de la función en $a$ evaluando $f(a)$. \\ 

\ex{} {
    Determine el resultado de los siguientes límites utilizando las propiedades 
    de los límites, justificando el procedimiento. \\

    \begin{enumerate}
        \item $\lim_{x \to 2} (2x^2 - 4x + 3)$ \\
            \begin{align*}
                \lim_{x \to 2} (2x^2 - 4x + 3) &= \lim_{x \to 2} (2x^2 - 4x) + \lim_{x \to 2} (3) \text{     (Propiedad 4)} \\
                &= \lim_{x \to 2} (2x^2 - 4x) + 3 \text{     (Propiedad 3)} \\
                &= \lim_{x \to 2} (2x^2) - \lim_{x \to 2} (4x) + 3 \text{     (Propiedad 5)} \\
                &= 2 \cdot \lim_{x \to 2} (x^2) - 4 \cdot \lim_{x \to 2} (x) + 3 \text{     (Propiedad 8)} \\
                &= 2 \cdot 2^2 - 4 \cdot 2 + 3 \text{     (Propiedad 3, 10)} \\
                &= 8 - 8 + 3 \\
                &= 3
            \end{align*}

        \item $\lim_{x \to -1} \frac{x^2+1}{x+2}$ \\
            \begin{align*}
                \lim_{x \to -1} \frac{x^2+1}{x+2} &= \frac{\lim_{x \to -1} (x^2+1)}{\lim_{x \to -1} (x+2)} \text{     (Propiedad 7)} \\
                &= \frac{\lim_{x \to -1} (x^2) + \lim_{x \to -1} (1)}{\lim_{x \to -1} (x) + \lim_{x \to -1} (2)} \text{     (Propiedad 4)} \\
                &= \frac{(-1)^2 + 1}{-1 + 2} \text{     (Propiedad 3, 10)} \\
                &= \frac{2}{1} \\
                &= 2
            \end{align*}
    \end{enumerate}

    \textbf{Nota:} Se hubiera obtenido el mismo resultado si se evaluaba directamente.
}

\section{Cálculo de límites}

Existen varias formas de calcular límites, las cuales se describen a continuación.

\subsection{Límites por sustitución directa}

A continuación se muestran algunos ejemplos de cómo se puede calcular el límite de una función 
utilizando la técnica de sustitución directa:

\ex{} {
    Determine mediante sustitución directa el límite de las siguientes funciones: \\

    \begin{enumerate}
        \item $\lim_{x \to 2} \sqrt{\frac{13x-1}{x^2}}$ \\
            \begin{align*}
                \lim_{x \to 2} \sqrt{\frac{13x-1}{x^2}} &= \sqrt{\frac{13(2)-1}{2^2}} \\
                &= \sqrt{\frac{25}{4}} \\
                &= \frac{5}{2} \\
            \end{align*}
        \item $\lim_{x \to 3} \frac{2x^2-1}{x^2+1}$ \\
            \begin{align*}
                \lim_{x \to 3} \frac{2x^2-1}{x^2+1} &= \frac{2(3)^2-1}{(3)^2+1} \\
                &= \frac{18-1}{9+1} \\
                &= \frac{17}{10} \\
            \end{align*}
    \end{enumerate}
}

\subsection{Forma indeterminada $\frac{0}{0}$}
