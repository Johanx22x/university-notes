\chapter{Guion presentacion individual}

\section{Introduccion}

\subsection{Saludo}

Buenas tardes publico presente, mi nombre es \textbf{Johan Rodriguez}.

Antes de comenzar me gustaria que se hagan una pregunta. Se han llegado 
a cuestionar en algun momento como se le podria llamar al conjunto de formas de 
aprender a traves de un medio electronico? Por ejemplo, un curso 
virtual o el mismo internet?

Bueno, en caso de que nunca les causara la suficiente curiosidad como para 
buscar acerca de ello, se los presento, se conoce como \textbf{e-learning}.

Pero, que es el e-learning exactamente, porque lo que acabo de mencionar 
puede llegar a ser muy ambiguo.

Ese es el tema del cual les hablare el dia de hoy.

\subsection{Agenda}

En este caso hablare sobre que es el e-learning exactamente, donde se 
encuentra aplicado y donde puede aplicarse y por ultimo cuales son las 
ventajas y desventajas de este tipo de aprendizaje.

\section{Que es el e-learning}

La universidad de Oxford define el e-learning como \textit{el aprendizaje
que se lleva a cabo a traves de medios electronicos, como internet,
computadoras, etc.}. 

Por otro lado, en un informe reciente del \textit{Canadian Council on Learning}
se menciona que el e-learning es \textit{el desarrollo de conocimientos y
habilidades a traves del uso de tecnologias de informacion y comunicacion.}

En resumen, el e-learning es un metodo de aprendizaje que se logra a traves
de las tecnologias que nos ofrece el siglo actual, siendo capaces de obtener,
visualizar, procesar y compartir informacion de manera rapida y eficiente, en 
este caso referente a la educacion.

\section{Aplicaciones del e-learning}

\subsection{Planteamiento}

Una vez que ya sabemos que es el e-learning, podemos preguntarnos donde se 
encuentra aplicado y donde se puede aplicar. O mejor dicho, donde se es util 
utilizar el e-learning.

La respuesta a esta pregunta es bastante sencilla, en cualquier lugar donde 
se pueda aplicar la tecnologia, se puede aplicar el e-learning. Ya que, como 
ya se menciono, el e-learning surge gracias a la tecnologia. Esto quiere decir 
que, en donde sea que exista un dispositivo tecnologico, e incluso internet,
se puede aplicar el e-learning.

A continuacion, les presentare algunos ejemplos de donde se puede aplicar este 
metodo de aprendizaje.

\subsection{Ejemplos}

Uno de los ejemplos mas comunes de aplicacion del e-learning es en los 
cursos virtuales. Estos cursos son impartidos por una institucion educativa, 
una persona en particular o inclusive por una empresa. 
En el formato sincronico, el curso se imparte en tiempo real, es decir, el alumno 
se encuentra en un ambiente virtual, en el cual puede interactuar con otros 
alumnos y con el profesor, asi como tambien puede acceder a material de estudio 
otorgado por el profesor. 
Tambien existe el formato asincronico, es decir, que no se tenga que estar 
conectado en un horario especifico, sino que se pueda acceder a el material de 
estudio en cualquier momento.
La mayor ventaja de estos cursos, es que por lo general se categorizan como 
aprendizaje formal, es decir, que se puede obtener un certificado o diploma por 
el curso.

Otro formato de e-learning es el del contenido multimedia. Este formato 
consiste en la creacion de contenidos multimedia, como videos, audios, 
presentaciones, etc. de los cuales el alumno se puede beneficiar durante el 
proceso de aprendizaje. Ejemplos de este metodo son los videos de Youtube,
los audios de Spotify, las presentaciones de Slideshare, etc. Este metodo 
es comunmente utilizado en cursos virtuales, pero tambien puede ser utilizado 
de manera independiente. Por decirlo de alguna manera, es como si el alumno 
se convirtiera en su propio profesor, ya que el mismo se encarga de buscar 
informacion y material de estudio que le pueda ser util. Un ejemplo de ello 
es el uso de Youtube para aprender a tocar un instrumento musical.

El ultimo formato que les presentare es el del contenido textual. Este 
formato consiste en la creacion de contenidos escritos, como libros, 
articulos, blogs, documentos, etc. Al igual que el contenido multimedia, este 
metodo tambien es comunmente utilizado en cursos virtuales. Otra caracteristica 
es que este metodo puede apoyarse con el contenido multimedia, ya que se 
puede utilizar para complementar la informacion que se encuentra en el 
contenido textual. Un ejemplo de ello es el uso de libros de texto para 
aprender una materia en especifico, o el uso de blogs para aprender a 
programar.

Existen muchos mas formatos de e-learning los cuales me gustaria presentarles,
pero en general los demas se encuentran formados por una combinacion de los 
formatos que ya les mencione, lo cual seria redundante mencionar.

En general las aplicaciones del e-learning son muy amplias, pero se pueden 
definir en dos grandes categorias, las orientadas a la educacion y las 
orientadas al fortalecimiento de las habilidades.

La educacion es la que generalmente conocemos, es decir, alumnado y profesor.

Por otro lado, el fortalecimiento de las habilidades se refiere a la 
capacitacion de personas en un area especifica, ya sea para mejorar sus 
habilidades o para aprender algo nuevo.

Esta ultima categoria es comunmente utilizada por empresas, ya que les
permite capacitar a sus empleados en un area especifica.

\section{Ventajas y desventajas}

\subsection{Planteamiento}

Ahora que ya sabemos que es el e-learning y donde se puede aplicar, podemos 
preguntarnos cuales son las ventajas y desventajas de este metodo de 
aprendizaje.

\subsection{Ventajas}

Las ventajas del e-learning son muchas, pero las mas importantes son las 
siguientes: 

\begin{itemize}
    \item El e-learning es un metodo de aprendizaje el cual otorga conocimiento 
        de manera rapida y eficiente.

    \item Ofrece comodidad al alumno, ya que no tiene que desplazarse a un lugar 
        especifico para recibir el conocimiento. Ademas, siendo flexible, por 
        lo que se puede acceder a el material de estudio en cualquier lugar y 
        momento.

    \item El e-learning se encuentra globalizado, ya que puede ser aplicado en 
        cualquier parte del mundo y por cualquier persona en cualquier momento.

    \item Ofrece inclusividad, esto gracias a la tecnologia, ya que esta permite 
        que cualquier persona pueda acceder al conocimiento, sin importar su 
        condicion fisica o mental. Inclusive ofrece la posibilidad de que personas 
        con discapacidad puedan acceder al conocimiento de manera mas facil.

    \item El e-learning es un metodo de aprendizaje el cual ofrece una gran 
        cantidad de oportunidades, teniendo el potencial de cambiar la forma en 
        la que se aprende en el mundo.

\end{itemize}

\subsection{Desventajas}

Las desventajas del e-learning son pocas, pero las mas importantes son las 
siguientes: 

\begin{itemize}
    \item El e-learning puede ser costoso, ya que requiere de una gran cantidad 
        de recursos tecnologicos y humanos.

    \item El e-learning es un metodo de aprendizaje el cual puede ser dificil de 
        implementar, ya que requiere una gran cantidad de disciplina y 
        compromiso por parte del alumno.

    \item La gran mayoria de las veces se requiere de acceso a internet, lo cual 
        puede ser un problema para personas que no cuentan con este servicio.

    \item La exposicion a la tecnologia puede ser perjudicial para el desarrollo 
        de las habilidades sociales de los alumnos.
    
    \item El uso de una pantalla electronica puede ser perjudicial para la 
        salud visual de las personas en general.

\end{itemize}

\section{Conclusion}

En conclusion, el e-learning es un metodo de aprendizaje el cual ofrece una gran 
cantidad de ventajas, pero tambien tiene sus desventajas. Sin embargo, 
considero que el e-learning es un metodo de aprendizaje el cual puede ser 
utilizado para mejorar la educacion en el mundo, ya que ofrece una gran 
cantidad de oportunidades a todas aquellos que deseen aprender.

Por ultimo, me gustaria mencionar una frase de David Warlick, quien es un 
educador y escritor estadounidense: \textit{``Necesitamos la tecnologia en 
cada aula y en las manos de cada estudiante y de cada profesor, porque 
es el boligrafo y el papel de nuestro tiempo y es la lente a traves de la cual 
experimentamos gran parte de nuestro mundo''}.

Muchas gracias por su atencion, y espero que esta presentacion les haya sido 
de utilidad. 

Si tienen alguna pregunta, estare encantado de responderla.
